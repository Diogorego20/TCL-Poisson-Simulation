\documentclass[12pt,a4paper]{article}

% Pacotes necessários
\usepackage[utf8]{inputenc}
\usepackage[portuguese]{babel}
\usepackage[T1]{fontenc}
\usepackage{amsmath}
\usepackage{amsfonts}
\usepackage{amssymb}
\usepackage{amsthm}
\usepackage{graphicx}
\usepackage{float}
\usepackage{booktabs}
\usepackage{array}
\usepackage{multirow}
\usepackage{longtable}
\usepackage{xcolor}
\usepackage{tikz}
\usepackage{pgfplots}
\usepackage{listings}
\usepackage{fancyhdr}
\usepackage{geometry}
\usepackage{hyperref}
\usepackage{enumitem}
\usepackage{tcolorbox}
\usepackage{mdframed}

% Configuração da página
\geometry{
    left=2.5cm,
    right=2.5cm,
    top=3cm,
    bottom=3cm
}

% Configuração do hyperref
\hypersetup{
    colorlinks=true,
    linkcolor=blue,
    filecolor=magenta,      
    urlcolor=cyan,
    citecolor=red
}

% Definição de cores
\definecolor{azulprincipal}{RGB}{52,152,219}
\definecolor{azulescuro}{RGB}{44,62,80}
\definecolor{verdeclaro}{RGB}{46,204,113}
\definecolor{cinzaclaro}{RGB}{245,245,245}
\definecolor{laranjaclaro}{RGB}{255,165,0}

% Configuração do pgfplots
\pgfplotsset{compat=1.18}

% Configuração de caixas coloridas
\newtcolorbox{definicao}{
    colback=azulprincipal!10,
    colframe=azulprincipal,
    title=Definição,
    fonttitle=\bfseries
}

\newtcolorbox{exemplo}{
    colback=verdeclaro!10,
    colframe=verdeclaro,
    title=Exemplo,
    fonttitle=\bfseries
}

\newtcolorbox{importante}{
    colback=laranjaclaro!10,
    colframe=laranjaclaro,
    title=Importante,
    fonttitle=\bfseries
}

\newtcolorbox{aplicacao}{
    colback=cinzaclaro,
    colframe=azulescuro,
    title=Aplicação Prática,
    fonttitle=\bfseries
}

% Configuração do listings para código R
\lstset{
    language=R,
    basicstyle=\ttfamily\footnotesize,
    keywordstyle=\color{blue}\bfseries,
    commentstyle=\color{green!60!black},
    stringstyle=\color{red},
    numbers=left,
    numberstyle=\tiny\color{gray},
    stepnumber=1,
    numbersep=5pt,
    backgroundcolor=\color{gray!10},
    frame=single,
    rulecolor=\color{gray!30},
    breaklines=true,
    breakatwhitespace=false,
    showspaces=false,
    showstringspaces=false,
    showtabs=false,
    tabsize=2
}

% Configuração de cabeçalho e rodapé
\pagestyle{fancy}
\fancyhf{}
\fancyhead[L]{Guia de Cálculos e Interpretação de Gráficos}
\fancyhead[R]{Teorema Central do Limite}
\fancyfoot[C]{\thepage}

% Informações do documento
\title{\textbf{GUIA EXPLICATIVO: CÁLCULOS E INTERPRETAÇÃO DE GRÁFICOS}\\
\large Teorema Central do Limite - Simulação de Monte Carlo}
\author{Material Didático Complementar}
\date{\today}

\begin{document}

\maketitle

\tableofcontents
\newpage

% ============================================================================
\section{Introdução}
% ============================================================================

Este documento apresenta uma explicação detalhada dos cálculos matemáticos e interpretação dos gráficos utilizados na demonstração do Teorema Central do Limite através de simulação de Monte Carlo. O material foi desenvolvido para facilitar a compreensão dos conceitos estatísticos e sua aplicação prática no contexto empresarial de marketing digital.

\begin{importante}
Este guia complementa a apresentação principal e fornece o embasamento matemático necessário para compreender completamente os resultados obtidos na simulação.
\end{importante}

% ============================================================================
\section{Conceitos Fundamentais}
% ============================================================================

\subsection{Distribuição de Probabilidade}

\begin{definicao}
Uma \textbf{distribuição de probabilidade} descreve como os valores de uma variável aleatória estão distribuídos no espaço amostral. No contexto do nosso estudo, analisamos o número de notificações que um influenciador recebe por minuto durante uma transmissão ao vivo.
\end{definicao}

\subsection{Parâmetros Importantes}

\begin{itemize}
    \item \textbf{$\lambda$ (lambda) = 6}: Parâmetro da distribuição de Poisson
    \begin{itemize}
        \item Representa a \textbf{média} de notificações por minuto
        \item Também representa a \textbf{variância} da distribuição original
    \end{itemize}
    
    \item \textbf{$n$}: Tamanho da amostra
    \begin{itemize}
        \item $n = 10$: observamos 10 minutos e calculamos a média
        \item $n = 30$: observamos 30 minutos e calculamos a média
        \item $n = 50$: observamos 50 minutos e calculamos a média
        \item $n = 100$: observamos 100 minutos e calculamos a média
    \end{itemize}
\end{itemize}

% ============================================================================
\section{Distribuição de Poisson}
% ============================================================================

% SEÇÃO 3.1 REMOVIDA conforme solicitado
% SEÇÃO 3.2 REMOVIDA conforme solicitado

\subsection{Propriedades da Distribuição de Poisson}

Para uma distribuição de Poisson com parâmetro $\lambda$:
\begin{itemize}
    \item \textbf{Média}: $E[X] = \lambda = 6$
    \item \textbf{Variância}: $\text{Var}(X) = \lambda = 6$
    \item \textbf{Desvio Padrão}: $\sigma = \sqrt{\lambda} = \sqrt{6} \approx 2.45$
\end{itemize}

% ============================================================================
\section{Teorema Central do Limite}
% ============================================================================

\subsection{Enunciado do Teorema}

\begin{definicao}
O \textbf{Teorema Central do Limite (TCL)} estabelece que, independentemente da forma da distribuição original, a distribuição das médias amostrais se aproxima de uma distribuição normal quando o tamanho da amostra aumenta, desde que as observações sejam independentes e identicamente distribuídas.
\end{definicao}

Matematicamente, se $X_1, X_2, \ldots, X_n$ são variáveis aleatórias independentes e identicamente distribuídas com média $\mu$ e variância $\sigma^2$, então:

\begin{equation}
\frac{\bar{X} - \mu}{\sigma/\sqrt{n}} \xrightarrow{d} N(0,1) \quad \text{quando } n \to \infty
\end{equation}

onde $\bar{X} = \frac{1}{n}\sum_{i=1}^{n} X_i$ é a média amostral.

\subsection{Aplicação ao Nosso Caso}

No contexto do nosso estudo:
\begin{itemize}
    \item \textbf{Distribuição original}: Poisson($\lambda = 6$) - assimétrica
    \item \textbf{Distribuição das médias}: Normal($\mu = 6, \sigma^2 = 6/n$) - simétrica
    \item \textbf{Convergência}: Quanto maior $n$, mais próxima da normal
\end{itemize}

% ============================================================================
\section{Cálculos Teóricos}
% ============================================================================

\subsection{Valor Esperado da Média Amostral}

\begin{definicao}
O valor esperado da média amostral é:
\begin{equation}
E[\bar{X}] = E[X] = \lambda = 6
\end{equation}
\end{definicao}

\begin{importante}
Este resultado é fundamental: independentemente do tamanho da amostra, a média das médias amostrais sempre será igual à média populacional ($\lambda = 6$).
\end{importante}

% SEÇÃO 4.2 REMOVIDA conforme solicitado

% ============================================================================
\section{Interpretação dos Gráficos}
% ============================================================================

\subsection{Gráfico da Distribuição de Poisson}

\subsubsection{Descrição}
Este gráfico apresenta a função de probabilidade da distribuição de Poisson com $\lambda = 6$.

\subsubsection{Como Interpretar}
\begin{itemize}
    \item \textbf{Eixo X}: Número de notificações ($k = 0, 1, 2, \ldots, 15$)
    \item \textbf{Eixo Y}: Probabilidade $P(X = k)$
    \item \textbf{Barras altas}: Valores mais prováveis (5, 6, 7 notificações)
    \item \textbf{Barras baixas}: Valores raros (0, 1, 12+ notificações)
    \item \textbf{Formato}: Assimétrico, com cauda à direita
\end{itemize}

\begin{aplicacao}
"Em qualquer minuto da live, é mais provável receber 6 notificações, mas pode variar de 0 a 15+. Valores entre 4-8 ocorrem em aproximadamente 70\% das vezes."
\end{aplicacao}

\subsection{Gráfico da Variância Teórica}

\subsubsection{Descrição}
Este gráfico mostra como a variância da média amostral diminui com o aumento do tamanho da amostra.

\subsubsection{Como Interpretar}
\begin{itemize}
    \item \textbf{Eixo X}: Tamanho da amostra ($n = 10, 30, 50, 100$)
    \item \textbf{Eixo Y}: Variância ($6/n$)
    \item \textbf{Tendência}: Decrescimento hiperbólico
    \item \textbf{Interpretação}: Relação inversa entre $n$ e variância
\end{itemize}

\begin{aplicacao}
"Lives mais longas têm resultados mais previsíveis. A variância diminui proporcionalmente ao inverso do tempo de observação."
\end{aplicacao}

% SEÇÃO 6.2.2 REMOVIDA conforme solicitado

\subsection{Gráfico de Convergência para a Normal}

\subsubsection{Descrição}
Gráfico de linhas mostrando as curvas normais teóricas para cada tamanho de amostra.

\subsubsection{Como Interpretar}
\begin{itemize}
    \item \textbf{Eixo X}: Valor da média amostral
    \item \textbf{Eixo Y}: Densidade de probabilidade
    \item \textbf{Linha azul ($n=10$)}: Curva mais "achatada" e larga
    \item \textbf{Linha verde ($n=30$)}: Curva mais alta e estreita
    \item \textbf{Linha vermelha ($n=100$)}: Curva muito alta e muito estreita
    \item \textbf{Linha vertical}: Marca $\lambda = 6$ (centro de todas as curvas)
\end{itemize}

\begin{aplicacao}
"Este gráfico demonstra matematicamente por que lives longas são mais previsíveis. A curva vermelha ($n=100$) é muito concentrada em 6, garantindo resultados consistentes."
\end{aplicacao}

% ============================================================================
\section{Simulação de Monte Carlo}
% ============================================================================

\subsection{Metodologia}

A simulação de Monte Carlo é um método computacional que utiliza amostragem aleatória repetida para resolver problemas matemáticos complexos.

% SEÇÃO 7.1.1 REMOVIDA conforme solicitado

\subsection{Justificativa do Número de Simulações}

\begin{importante}
\textbf{Por que 1000 repetições?}

A Lei dos Grandes Números garante que quanto mais repetições, mais próximo chegamos do valor teórico:
\begin{itemize}
    \item 10 repetições: Resultado impreciso
    \item 100 repetições: Resultado razoável
    \item 1000 repetições: Resultado muito confiável
    \item 10000 repetições: Resultado extremamente preciso (mas computacionalmente custoso)
\end{itemize}
\end{importante}

% ============================================================================
\section{Análise dos Resultados}
% ============================================================================

\subsection{Tabela de Resultados da Simulação}

\begin{table}[H]
\centering
\caption{Comparação entre Valores Teóricos e Observados}
\begin{tabular}{@{}cccccc@{}}
\toprule
\textbf{$n$} & \textbf{Média Teórica} & \textbf{Média Observada} & \textbf{Erro (\%)} & \textbf{Var. Teórica} & \textbf{Var. Observada} \\
\midrule
10  & 6.0000 & 5.9704 & 0.49 & 0.6000 & 0.5662 \\
30  & 6.0000 & 5.9739 & 0.44 & 0.2000 & 0.2039 \\
50  & 6.0000 & 6.0041 & 0.07 & 0.1200 & 0.1130 \\
100 & 6.0000 & 6.0072 & 0.12 & 0.0600 & 0.0605 \\
\bottomrule
\end{tabular}
\end{table}

\subsection{Interpretação dos Resultados}

\subsubsection{Convergência da Média}
\begin{itemize}
    \item Todas as médias observadas ficaram muito próximas de 6
    \item Erro máximo foi de apenas 0.49\%
    \item Confirma que $E[\bar{X}] = \lambda = 6$
\end{itemize}

\subsubsection{Comportamento da Variância}
\begin{itemize}
    \item Variância diminui conforme $n$ aumenta
    \item Valores observados muito próximos dos teóricos
    \item Confirma que $\text{Var}(\bar{X}) = \lambda/n$
\end{itemize}

\subsubsection{Verificação do TCL}
\begin{itemize}
    \item Histogramas mostram aproximação à normal
    \item Quanto maior $n$, mais "normal" fica a distribuição
    \item TCL foi verificado experimentalmente
\end{itemize}

% SEÇÃO 9 REMOVIDA conforme solicitado (Exercícios Práticos)
% SEÇÃO 10 REMOVIDA conforme solicitado (Conclusões)
% SEÇÃO 11 REMOVIDA conforme solicitado (Referências)

\end{document}

